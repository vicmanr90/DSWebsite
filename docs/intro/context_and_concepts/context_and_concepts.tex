% Options for packages loaded elsewhere
\PassOptionsToPackage{unicode}{hyperref}
\PassOptionsToPackage{hyphens}{url}
%
\documentclass[
  ignorenonframetext,
]{beamer}
\usepackage{pgfpages}
\setbeamertemplate{caption}[numbered]
\setbeamertemplate{caption label separator}{: }
\setbeamercolor{caption name}{fg=normal text.fg}
\beamertemplatenavigationsymbolsempty
% Prevent slide breaks in the middle of a paragraph
\widowpenalties 1 10000
\raggedbottom
\setbeamertemplate{part page}{
  \centering
  \begin{beamercolorbox}[sep=16pt,center]{part title}
    \usebeamerfont{part title}\insertpart\par
  \end{beamercolorbox}
}
\setbeamertemplate{section page}{
  \centering
  \begin{beamercolorbox}[sep=12pt,center]{part title}
    \usebeamerfont{section title}\insertsection\par
  \end{beamercolorbox}
}
\setbeamertemplate{subsection page}{
  \centering
  \begin{beamercolorbox}[sep=8pt,center]{part title}
    \usebeamerfont{subsection title}\insertsubsection\par
  \end{beamercolorbox}
}
\AtBeginPart{
  \frame{\partpage}
}
\AtBeginSection{
  \ifbibliography
  \else
    \frame{\sectionpage}
  \fi
}
\AtBeginSubsection{
  \frame{\subsectionpage}
}

\usepackage{amsmath,amssymb}
\usepackage{iftex}
\ifPDFTeX
  \usepackage[T1]{fontenc}
  \usepackage[utf8]{inputenc}
  \usepackage{textcomp} % provide euro and other symbols
\else % if luatex or xetex
  \usepackage{unicode-math}
  \defaultfontfeatures{Scale=MatchLowercase}
  \defaultfontfeatures[\rmfamily]{Ligatures=TeX,Scale=1}
\fi
\usepackage{lmodern}
\usetheme[]{CambridgeUS}
\ifPDFTeX\else  
    % xetex/luatex font selection
\fi
% Use upquote if available, for straight quotes in verbatim environments
\IfFileExists{upquote.sty}{\usepackage{upquote}}{}
\IfFileExists{microtype.sty}{% use microtype if available
  \usepackage[]{microtype}
  \UseMicrotypeSet[protrusion]{basicmath} % disable protrusion for tt fonts
}{}
\makeatletter
\@ifundefined{KOMAClassName}{% if non-KOMA class
  \IfFileExists{parskip.sty}{%
    \usepackage{parskip}
  }{% else
    \setlength{\parindent}{0pt}
    \setlength{\parskip}{6pt plus 2pt minus 1pt}}
}{% if KOMA class
  \KOMAoptions{parskip=half}}
\makeatother
\usepackage{xcolor}
\newif\ifbibliography
\setlength{\emergencystretch}{3em} % prevent overfull lines
\setcounter{secnumdepth}{-\maxdimen} % remove section numbering


\providecommand{\tightlist}{%
  \setlength{\itemsep}{0pt}\setlength{\parskip}{0pt}}\usepackage{longtable,booktabs,array}
\usepackage{calc} % for calculating minipage widths
\usepackage{caption}
% Make caption package work with longtable
\makeatletter
\def\fnum@table{\tablename~\thetable}
\makeatother
\usepackage{graphicx}
\makeatletter
\def\maxwidth{\ifdim\Gin@nat@width>\linewidth\linewidth\else\Gin@nat@width\fi}
\def\maxheight{\ifdim\Gin@nat@height>\textheight\textheight\else\Gin@nat@height\fi}
\makeatother
% Scale images if necessary, so that they will not overflow the page
% margins by default, and it is still possible to overwrite the defaults
% using explicit options in \includegraphics[width, height, ...]{}
\setkeys{Gin}{width=\maxwidth,height=\maxheight,keepaspectratio}
% Set default figure placement to htbp
\makeatletter
\def\fps@figure{htbp}
\makeatother

\useoutertheme{infolines}
\useinnertheme{rectangles}
\AtBeginSection[]
{
\begin{frame}<beamer>
    \frametitle{Outline %for section \thesection}
    \tableofcontents[currentsection]
\end{frame}
}
\makeatletter
\makeatother
\makeatletter
\makeatother
\makeatletter
\@ifpackageloaded{caption}{}{\usepackage{caption}}
\AtBeginDocument{%
\ifdefined\contentsname
  \renewcommand*\contentsname{Table of contents}
\else
  \newcommand\contentsname{Table of contents}
\fi
\ifdefined\listfigurename
  \renewcommand*\listfigurename{List of Figures}
\else
  \newcommand\listfigurename{List of Figures}
\fi
\ifdefined\listtablename
  \renewcommand*\listtablename{List of Tables}
\else
  \newcommand\listtablename{List of Tables}
\fi
\ifdefined\figurename
  \renewcommand*\figurename{Figure}
\else
  \newcommand\figurename{Figure}
\fi
\ifdefined\tablename
  \renewcommand*\tablename{Table}
\else
  \newcommand\tablename{Table}
\fi
}
\@ifpackageloaded{float}{}{\usepackage{float}}
\floatstyle{ruled}
\@ifundefined{c@chapter}{\newfloat{codelisting}{h}{lop}}{\newfloat{codelisting}{h}{lop}[chapter]}
\floatname{codelisting}{Listing}
\newcommand*\listoflistings{\listof{codelisting}{List of Listings}}
\makeatother
\makeatletter
\@ifpackageloaded{caption}{}{\usepackage{caption}}
\@ifpackageloaded{subcaption}{}{\usepackage{subcaption}}
\makeatother
\makeatletter
\@ifpackageloaded{tcolorbox}{}{\usepackage[skins,breakable]{tcolorbox}}
\makeatother
\makeatletter
\@ifundefined{shadecolor}{\definecolor{shadecolor}{rgb}{.97, .97, .97}}
\makeatother
\makeatletter
\makeatother
\makeatletter
\makeatother
\ifLuaTeX
  \usepackage{selnolig}  % disable illegal ligatures
\fi
\IfFileExists{bookmark.sty}{\usepackage{bookmark}}{\usepackage{hyperref}}
\IfFileExists{xurl.sty}{\usepackage{xurl}}{} % add URL line breaks if available
\urlstyle{same} % disable monospaced font for URLs
\hypersetup{
  pdftitle={Contextualization and contextualization},
  pdfauthor={Víctor Mario Noble Ramos, MSc.},
  hidelinks,
  pdfcreator={LaTeX via pandoc}}

\title{Contextualization and contextualization}
\author{Víctor Mario Noble Ramos, MSc.}
\date{2024-08-29}

\begin{document}
\frame{\titlepage}
\ifdefined\Shaded\renewenvironment{Shaded}{\begin{tcolorbox}[sharp corners, breakable, enhanced, boxrule=0pt, interior hidden, frame hidden, borderline west={3pt}{0pt}{shadecolor}]}{\end{tcolorbox}}\fi

\hypertarget{course-presentation}{%
\section{Course presentation}\label{course-presentation}}

\begin{frame}{Course presentation}
\protect\hypertarget{course-presentation-1}{}
\begin{block}{Welcome to introduction to Machine Learning: algorithms
and theory!}
\protect\hypertarget{welcome-to-introduction-to-machine-learning-algorithms-and-theory}{}
\end{block}

\begin{alertblock}{Today's sesion's purposes}
\protect\hypertarget{todays-sesions-purposes}{}
\begin{itemize}[<+->]
\tightlist
\item
  To meet each other.
\item
  To explain the course objective, context and contents.
\item
  To explore the resources we will be using.
\item
  To know how the evaluation will be done.
\item
  To start!
\end{itemize}
\end{alertblock}
\end{frame}

\begin{frame}{Presentation}
\protect\hypertarget{presentation}{}
\begin{exampleblock}{Victor Mario Noble Ramos}
\protect\hypertarget{victor-mario-noble-ramos}{}
\begin{itemize}
\tightlist
\item
  Industrial engineer with a master in Production Engineering from
  \emph{Universidade Federal de São Carlos} - UFSCar.
\item
  +10 years of experience in teaching and academic research.
\item
  Graduated with honors: DS4A - Correlation One \& MINTIC
\item
  Research topics: Operations research, Data Science, Engineering
  education.
\end{itemize}

\pause
\end{exampleblock}

\begin{block}{}
\protect\hypertarget{section}{}
\emph{What about you?}
\end{block}
\end{frame}

\begin{frame}{Course content}
\protect\hypertarget{course-content}{}
\textbf{Course objective}:

\begin{alertblock}{Goal}
\protect\hypertarget{goal}{}
To provide students with a solid understanding of the fundamental
principles of machine learning, from data preparation and cleaning to
the implementation and evaluation of basic models.

\pause
\end{alertblock}

\begin{exampleblock}{Macrounits}
\protect\hypertarget{macrounits}{}
\begin{enumerate}
\tightlist
\item
  Generalities of the field
\item
  Data preparation, EDA and visualization
\item
  ML models
\item
  Common challenges
\end{enumerate}
\end{exampleblock}
\end{frame}

\begin{frame}{1. Generalities of the field}
\protect\hypertarget{generalities-of-the-field}{}
\begin{alertblock}{Goal}
\protect\hypertarget{goal-1}{}
To know and explore different concepts, history and the context of the
Data Sciences and the Machine Learning field, as well as to refresh
previous notions of the discipline.

\pause
\end{alertblock}

\begin{exampleblock}{Contents 1/3}
\protect\hypertarget{contents-13}{}
\begin{itemize}
\tightlist
\item
  Concepts: Machine learning, Statistical learning, Deep learning,
  Artificial intelligence, Data science, etc.
\item
  Importance and history of the field.
\item
  Applications
\end{itemize}
\end{exampleblock}
\end{frame}

\begin{frame}{1. Generalities of the field}
\protect\hypertarget{generalities-of-the-field-1}{}
\begin{alertblock}{Goal}
\protect\hypertarget{goal-2}{}
To know and explore different concepts, history and the context of the
Data Sciences and the Machine Learning field, as well as to refresh
previous notions of the discipline.
\end{alertblock}

\begin{exampleblock}{Contents 2/3}
\protect\hypertarget{contents-23}{}
\begin{itemize}
\tightlist
\item
  Refresher of statistics

  \begin{itemize}
  \tightlist
  \item
    Random variables and data types
  \item
    Descriptive statistics
  \item
    Probability distributions
  \end{itemize}
\end{itemize}
\end{exampleblock}
\end{frame}

\begin{frame}{1. Generalities of the field}
\protect\hypertarget{generalities-of-the-field-2}{}
\begin{alertblock}{Goal}
\protect\hypertarget{goal-3}{}
To know and explore different concepts, history and the context of the
Data Sciences and the Machine Learning field, as well as to refresh
previous notions of the discipline.
\end{alertblock}

\begin{exampleblock}{Contents 3/3}
\protect\hypertarget{contents-33}{}
\begin{itemize}
\tightlist
\item
  Refresher of programming

  \begin{itemize}
  \tightlist
  \item
    Data structures: the dataframe, series
  \item
    Control flow statements
  \item
    Work with libraries (modules)
  \end{itemize}
\end{itemize}
\end{exampleblock}
\end{frame}

\begin{frame}{2. Data preparation, EDA and visualization}
\protect\hypertarget{data-preparation-eda-and-visualization}{}
\begin{alertblock}{Goal}
\protect\hypertarget{goal-4}{}
To teach to the students how to prepare and clean the data as well as
how to explore and understand the data before applying ML models.\pause
\end{alertblock}

\begin{exampleblock}{Contents - 2.1. Data preparation}
\protect\hypertarget{contents---2.1.-data-preparation}{}
\begin{itemize}
\tightlist
\item
  Introduction to pandas.
\item
  Data importing, storage and file formats.
\item
  Handling missing values, error detection, and correction.
\item
  Data normalization and standardization.
\item
  Feature engineering: creation and selection of features.
\end{itemize}
\end{exampleblock}
\end{frame}

\begin{frame}{2. Data preparation, EDA and visualization}
\protect\hypertarget{data-preparation-eda-and-visualization-1}{}
\begin{alertblock}{Goal}
\protect\hypertarget{goal-5}{}
To teach to the students how to prepare and clean the data as well as
how to explore and understand the data before applying ML models.
\end{alertblock}

\begin{exampleblock}{Contents - 2.2. Data wrangling}
\protect\hypertarget{contents---2.2.-data-wrangling}{}
\begin{itemize}
\tightlist
\item
  Importing and merging data from different tables {[}and query language
  between tables (Optional){]}.
\item
  Reshaping and pivoting.
\item
  String manipulation.
\end{itemize}
\end{exampleblock}
\end{frame}

\begin{frame}{2. Data preparation, EDA and visualization}
\protect\hypertarget{data-preparation-eda-and-visualization-2}{}
\begin{alertblock}{Goal}
\protect\hypertarget{goal-6}{}
To teach to the students how to prepare and clean the data as well as
how to explore and understand the data before applying ML models.
\end{alertblock}

\begin{exampleblock}{Contents - 2.3. EDA and visualization}
\protect\hypertarget{contents---2.3.-eda-and-visualization}{}
\begin{itemize}
\tightlist
\item
  Plotting with matplotlib
\item
  Other libraries for plotting
\item
  Descriptive univariate statistics and probability distributions.
\item
  Outliers identification and handling.
\item
  Bivariate and multivariate relations (Correlation matrix)
\item
  Dimensionality reduction.
\item
  Preliminary hypotheses testing.
\end{itemize}
\end{exampleblock}
\end{frame}

\begin{frame}{3. ML models}
\protect\hypertarget{ml-models}{}
\begin{alertblock}{Goal}
\protect\hypertarget{goal-7}{}
To comprehend and apply some of the ML/ST models used for supervised
learning in real and synthetic data.
\end{alertblock}

\begin{exampleblock}{Contents - 3.1. Supervised learning 1/3}
\protect\hypertarget{contents---3.1.-supervised-learning-13}{}
\begin{itemize}
\tightlist
\item
  Regression and classification problems

  \begin{itemize}
  \tightlist
  \item
    Overview and refresher of linear regression
  \item
    Overview and refresher of logistic regression
  \item
    Linear model selection and regularization
  \end{itemize}
\end{itemize}
\end{exampleblock}
\end{frame}

\begin{frame}{3. ML models}
\protect\hypertarget{ml-models-1}{}
\begin{alertblock}{Goal}
\protect\hypertarget{goal-8}{}
To comprehend and apply some of the ML/ST models used for supervised
learning in real and synthetic data.
\end{alertblock}

\begin{exampleblock}{Contents - 3.1. Supervised learning 2/3}
\protect\hypertarget{contents---3.1.-supervised-learning-23}{}
\begin{itemize}
\tightlist
\item
  Extensions of linear-based methods

  \begin{itemize}
  \tightlist
  \item
    Polynomial regression
  \item
    Step functions
  \item
    Regression splines
  \item
    Smoothing splines
  \end{itemize}
\end{itemize}
\end{exampleblock}
\end{frame}

\begin{frame}{3. ML models}
\protect\hypertarget{ml-models-2}{}
\begin{alertblock}{Goal}
\protect\hypertarget{goal-9}{}
To comprehend and apply some of the ML/ST models used for supervised
learning in real and synthetic data.
\end{alertblock}

\begin{exampleblock}{Contents - 3.1. Supervised learning 3/3}
\protect\hypertarget{contents---3.1.-supervised-learning-33}{}
\begin{itemize}
\tightlist
\item
  Tree-based methods
\item
  Support vector machines
\item
  Introduction to deep learning
\end{itemize}
\end{exampleblock}
\end{frame}

\begin{frame}{3. ML models}
\protect\hypertarget{ml-models-3}{}
\begin{alertblock}{Goal}
\protect\hypertarget{goal-10}{}
To comprehend and apply some of the ML/ST models used for supervised
learning in real and synthetic data.
\end{alertblock}

\begin{exampleblock}{Contents - 3.2. Unsupervised learning}
\protect\hypertarget{contents---3.2.-unsupervised-learning}{}
\begin{itemize}
\tightlist
\item
  Fundamentals and problem statement
\item
  Techniques

  \begin{itemize}
  \tightlist
  \item
    K-means clustering
  \item
    Hierarchical clustering
  \end{itemize}
\end{itemize}
\end{exampleblock}
\end{frame}

\begin{frame}{3. Common challenges and final remarks}
\protect\hypertarget{common-challenges-and-final-remarks}{}
\begin{alertblock}{Goal}
\protect\hypertarget{goal-11}{}
To comprehend and apply some of the ML/ST models used for supervised
learning in real and synthetic data.
\end{alertblock}

\begin{exampleblock}{Contents}
\protect\hypertarget{contents}{}
\begin{itemize}
\tightlist
\item
  High dimensional data
\item
  Unbalanced data
\item
  Unlabeled data
\item
  Overfitting
\item
  Closing remarks
\end{itemize}
\end{exampleblock}
\end{frame}

\begin{frame}{Evaluation}
\protect\hypertarget{evaluation}{}
The course will be evaluated by three types of activities:

\begin{itemize}
\tightlist
\item
  Homework socialization (40\%)
\item
  Midterm exam (30\%)
\item
  Final exam (30\%)
\end{itemize}
\end{frame}

\begin{frame}{Resources}
\protect\hypertarget{resources}{}
We will be using the follwing resources. All are free!

\begin{itemize}
\tightlist
\item
  Cintia (Unicor)
\item
  Google colab
\item
  Jupyter notebooks
\item
  Kaggle
\end{itemize}
\end{frame}

\begin{frame}{References}
\protect\hypertarget{references}{}
\fontsize{10pt}{5.2}\selectfont

\begin{itemize}
\item
  \textbf{James, G., Witten, D., Hastie, T., \& Tibshirani, R. (2021).
  \emph{An Introduction to Statistical Learning (2nd ed.)}. Springer.}
\item
  Hastie, T., Tibshirani, R., \& Friedman, J. (2009). \emph{The Elements
  of Statistical Learning}. Springer New York.
  https://doi.org/10.1007/978-0-387-84858-7
\item
  Géron, A. (2023). \emph{Hands-On Machine Learning with Scikit-Learn,
  Keras, and TensorFlow (3rd ed.)}. O'Reilly Media Inc.
\item
  \textbf{McKinney, W. (2021). \emph{Python for Data Analysis: Data
  Wrangling with Pandas, NumPy, and IPython (3rd ed.)}. O'Reilly Media.}
\item
  Vicente Cestero, E., \& Mateos Caballero, A. (2023).
  \emph{Inteligencia Artificial: Fundamentos matemáticos, algorítmicos y
  metodológicos.} España: Eloy Vicente Cestero. ISBN 978-84-09-46911-6.
\item
  Berzal, F. (2018). \emph{Redes Neuronales \& Deep Learning.} España.
  Fernando Berzal. ISBN 1-7312-6538-7
\end{itemize}
\end{frame}

\hypertarget{let-us-start-fundamental-concepts}{%
\section{Let us start: Fundamental
concepts}\label{let-us-start-fundamental-concepts}}

\begin{frame}{Supervised and unsupervised learning}
\protect\hypertarget{supervised-and-unsupervised-learning}{}
\end{frame}

\begin{frame}{Unsupervised learning}
\protect\hypertarget{unsupervised-learning}{}
\end{frame}



\end{document}
